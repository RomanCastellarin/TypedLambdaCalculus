\documentclass[12pt]{article}
\usepackage[utf8]{inputenc}
\usepackage[spanish]{babel}
\usepackage[pdftex]{graphicx}
\usepackage{float}
\usepackage{booktabs}
\usepackage[table,xcdraw]{xcolor}
\usepackage{ltxtable}
\usepackage{proof}
\usepackage{amsmath}
\usepackage{ amssymb }
\usepackage[left=2cm, right=2cm, top=2cm]{geometry} 
\begin{document}
\begin{titlepage}

\begin{minipage}{2.6cm}
\includegraphics[width=\textwidth]{fceia.pdf}
\end{minipage}
\hfill
%
\begin{minipage}{6cm}
\begin{center}
\normalsize{Universidad Nacional de Rosario\\
Facultad de Ciencias Exactas,\\
Ingeniería y Agrimensura\\}
\end{center}
\end{minipage}
\hspace{0.5cm}
\hfill
\begin{minipage}{2.6cm}
\includegraphics[width=\textwidth]{unr.pdf}
\end{minipage}

\vspace{0.5cm}

\begin{center}
\normalsize{\sc Análisis de Lenguajes de Programación}\\
\vspace{0.5cm}
\large{Trabajo Práctico III}\\

\Large{\bf $\lambda$-cálculo tipado}\\
\vspace{5cm}

\normalsize
Román Castellarin\\
Juan Ignacio Suarez\\

\vspace*{0.5cm}
\small{ \today }


\end{center}
\end{titlepage}
\newpage

\subsection{Ejercicio 1}

Notemos con $B^2 \equiv (B \rightarrow B)$ y con $B^3 \equiv (B \rightarrow B \rightarrow B)$.

A continuación definimos los siguientes contextos:\\


$\Gamma_1 = \lbrace x:B^3, y:B^2, z:B \rbrace$

$\Gamma_2 = \lbrace x:B^3, y:B^2 \rbrace$

$\Gamma_3 = \lbrace x:B^3 \rbrace$\\

%%

Luego resulta,\\

\infer[\textsc{T-Abs}]{\vdash (\lambda x : B^3.\ \lambda y : B^2.\ \lambda z : B.\ (x\ z)\ (y\ z) ): B^3 \rightarrow B^2 \rightarrow B^2}{
	\infer[\textsc{T-Abs}]{\Gamma_3 \vdash (\lambda y : B^2.\ \lambda z : B.\ (x\ z)\ (y\ z) ): B^2 \rightarrow B^2}{
		\infer[\textsc{T-Abs}]{\Gamma_2 \vdash (\lambda z : B.\ (x\ z)\ (y\ z)) : B^2 }{
			\infer[\textsc{T-App}]{\Gamma_1 \vdash (x\ z)\ (y\ z) : B }{
				\infer[\textsc{T-App}]{\Gamma_1 \vdash x\ z : B^2 }{
					\infer[\textsc{T-Var}]{\Gamma_1 \vdash x : B^3 }{}
					&
					\infer[\textsc{T-Var}]{\Gamma_1 \vdash z : B }{}
				}
				&
				\infer[\textsc{T-App}]{\Gamma_1 \vdash y\ z : B }{
					\infer[\textsc{T-Var}]{\Gamma_1 \vdash y : B^2 }{}
					&
					\infer[\textsc{T-Var}]{\Gamma_1 \vdash z : B }{}
				}
			}
		}
	}
}


\subsection{Ejercicio 2}

De acuerdo a las reglas de tipado, algunas expresiones carecen de un tipo. \\
La función \textit{infer} devuelve \texttt{Either String Type} ya que si el término no está bien tipado se puede retornar un \texttt{String} con un mensaje de error.\\
Esto se combina a su vez con el operador \textit{bind} que toma un \texttt{Either String Type}, y una función \texttt{f} de \texttt{Type} en \texttt{Either String Type} y propaga el error si su primer argumento se trataba de un \texttt{String}, o en caso contrario aplica \texttt{f} al tipo contenido.\\
La utilidad de \textit{bind} se puede apreciar mejor cuando se encadenan varias funciones, dejando un código prolijo y sin repeticiones.

\subsection{Ejercicio 3 - 4}

\textit{(en el codigo fuente)}

\subsection{Ejercicio 5}

Conservando la notación anterior, tenemos\\

\infer[\textsc{T-Ascribe}]{ \vdash (\texttt{let }z=( \lambda x:B.\ x) \texttt{ as } B^2\texttt{ in } z)\texttt{ as } B^2 : B^2 }{
	\infer[\textsc{T-Let}]{ \vdash (\texttt{let }z=( \lambda x:B.\ x) \texttt{ as } B^2\texttt{ in } z) : B^2 }{
		\infer[\textsc{T-Ascribe}]{ \vdash( \lambda x:B.\ x) \texttt{ as } B^2 : B^2 }{
		    \infer[\textsc{T-Abs}]{ \vdash( \lambda x:B.\ x): B^2 }{
			    \infer[\textsc{T-Var}]{ x:B \vdash x : B }{}
			}
		}
		&
		\infer[\textsc{T-Var}]{ z:B^2 \vdash z : B^2 }{}
	}
}

\subsection{Ejercicio 6 - 8}

\textit{(en el codigo fuente)}

\subsection{Ejercicio 9}

Continuamos con la siguiente demostración:\\

\infer[\textsc{T-Fst}]{ \vdash \texttt{fst }(\texttt{unit as Unit}, \lambda x:(B,B).\ \texttt{snd } x) : \texttt{Unit}}{
	\infer[\textsc{T-Pair}]{ \vdash (\texttt{unit as Unit}, \lambda x:(B,B).\ \texttt{snd } x) : (\texttt{Unit}, (B,B) \rightarrow B) }{
		\infer[\textsc{T-Ascribe}]{ \vdash \texttt{unit as Unit} : \texttt{Unit} }{
		    \infer[\textsc{T-Unit}]{ \vdash \texttt{unit} : \texttt{Unit} }{}
		}
		&
		\infer[\textsc{T-Abs}]{ \vdash (\lambda x:(B,B).\ \texttt{snd } x) : (B,B) \rightarrow B }{
			\infer[\textsc{T-Snd}]{ x : (B,B) \vdash \texttt{snd } x : B }{
			    \infer[\textsc{T-Var}]{ x : (B,B) \vdash x : (B,B) }{}
			}
		}
	}
}

\subsection{Ejercicio 10}

\textit{(en el codigo fuente)}

\subsection{Ejercicio 11}

\textit{(en el archivo correspondiente)}

\end{document}